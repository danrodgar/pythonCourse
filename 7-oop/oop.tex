%%%%%%%%%%%%%%%%%%%%%%%%%%%%%%%%%%%%%%%%%%%%%%%%%%%%%%%%%%%%%%%%%
% MUW Presentation
% LaTeX Template
% Version 1.0 (27/12/2016)
%
% License:
% CC BY-NC-SA 4.0 (http://creativecommons.org/licenses/by-nc-sa/3.0/)
%
% Created by:
% Nicolas Ballarini, CeMSIIS, Medical University of Vienna
% nicoballarini@gmail.com
% http://statistics.msi.meduniwien.ac.at/
%
% Customized for UAH by:
% David F. Barrero, Departamento de Automática, UAH
%%%%%%%%%%%%%%%%%%%%%%%%%%%%%%%%%%%%%%%%%%%%%%%%%%%%%%%%%%%%%%%%%

\documentclass[10pt,compress]{beamer} % Change 10pt to make fonts of a different size
\mode<presentation>

\usepackage[spanish]{babel}
\usepackage{fontspec}
\usepackage{tikz}
\usepackage{etoolbox}
\usepackage{xcolor}
\usepackage{xstring}
\usepackage{listings}
\usepackage{multicol}
\usepackage{standalone}
\usepackage{tikz}
\usepackage{tikz-uml}
\usetikzlibrary{matrix,chains,positioning,decorations.pathreplacing,arrows,shapes}

\usetheme{UAH}
\usecolortheme{UAH}
\setbeamertemplate{navigation symbols}{} 
\setbeamertemplate{caption}[numbered]

%%%%%%%%%%%%%%%%%%%%%%%%%%%%%%%%%%%%%%%%%%%%%%%%%%%%%%%%%%%%%%%%%
%% Presentation Info
\title[OOP in Python]{Object-Oriented Programming in Python}
\author{\asignatura\\\carrera}
\institute{}
\date{Departamento de Automática}
%%%%%%%%%%%%%%%%%%%%%%%%%%%%%%%%%%%%%%%%%%%%%%%%%%%%%%%%%%%%%%%%%


%%%%%%%%%%%%%%%%%%%%%%%%%%%%%%%%%%%%%%%%%%%%%%%%%%%%%%%%%%%%%%%%%
%% Descomentar para habilitar barra de navegación superior
\setNavigation
%%%%%%%%%%%%%%%%%%%%%%%%%%%%%%%%%%%%%%%%%%%%%%%%%%%%%%%%%%%%%%%%%

%%%%%%%%%%%%%%%%%%%%%%%%%%%%%%%%%%%%%%%%%%%%%%%%%%%%%%%%%%%%%%%%%
%% Configuración de logotipos en portada
%% Opacidad de los logotipos
\newcommand{\opacidad}{1}
%% Descomentar para habilitar logotipo en pié de página de portada
\renewcommand{\logoUno}{Images/isg.png}
%% Descomentar para habilitar logotipo en pié de página de portada
%\renewcommand{\logoDos}{Images/CCLogo.png}
%% Descomentar para habilitar logotipo en pié de página de portada
%\renewcommand{\logoTres}{Images/ALogo.png}
%% Descomentar para habilitar logotipo en pié de página de portada
%\renewcommand{\logoCuatro}{Images/ELogo.png}
%%%%%%%%%%%%%%%%%%%%%%%%%%%%%%%%%%%%%%%%%%%%%%%%%%%%%%%%%%%%%%%%%

%%%%%%%%%%%%%%%%%%%%%%%%%%%%%%%%%%%%%%%%%%%%%%%%%%%%%%%%%%%%%%%%%
%% FOOTLINE
%% Comment/Uncomment the following blocks to modify the footline
%% content in the body slides. 


%% Option A: Title and institute
\footlineA
%% Option B: Author and institute
%\footlineB
%% Option C: Title, Author and institute
%\footlineC
%%%%%%%%%%%%%%%%%%%%%%%%%%%%%%%%%%%%%%%%%%%%%%%%%%%%%%%%%%%%%%%%%

\begin{document}

%%%%%%%%%%%%%%%%%%%%%%%%%%%%%%%%%%%%%%%%%%%%%%%%%%%%%%%%%%%%%%%%%
% Use this block for a blue title slide with modified footline
{\titlepageBlue
    \begin{frame}
        \titlepage
    \end{frame}
}

\institute{\asignatura}

\begin{frame}[plain]{}
	\begin{block}{Objectives}
		\begin{enumerate}
		\item Introduce basic programming concepts.
		\item Understand the main characteristics of Object-Oriented Programming (OOP).
		\item Use Python to implement class hierarchies
		\item Use class libraries
		\end{enumerate}
	\end{block}
\end{frame}


{
\disableNavigation{white}
\begin{frame}[shrink]{Table of Contents}
 \frametitle{Table of Contents}
    \begin{multicols}{2}
        \tableofcontents
    \end{multicols}
  % You might wish to add the option [pausesections]
\end{frame}
}

\section[Programming paradigms]{Programming paradigms}

\subsection[Understanding concepts]{Understanding concepts}

\begin{frame}{Understanding concepts}{Differentiate between \ldots}
			\begin{block}{Programming}
				Set of techniques that allow the development of programs using a programming language.
			\end{block}
			\begin{block}{Programming language}
				Set of rules and instructions based on a familiar syntax and later translated into machine language which allow the elaboration of a program to solve a problem.			\end{block}
			\begin{block}{Paradigm}
Set of rules, patterns and styles of programming that are used by programming languages.
%	Conjunto de estandares, teorias y metodos para representar un modo de organizar el pensamiento. % Forma de entender la realidad. 
			\end{block}
\end{frame}

\subsection[Programming paradigms types]{Programming paradigms types}

\begin{frame}{Programming paradigms types (I)}{}
	\begin{block}{Declarative programming}
Describe \alert{what} is used to calculate through conditions, propositions, statements, etc., but does not specify \alert{how}.
  	\end{block}
  	\begin{itemize}
  		\item \textbf{Logic}: follows the first order predicate logic in order to formalize facts of the real world. 								  \textit{(Prolog)}
  		\begin{itemize}
  			\item Example: \textit{Anne's father is Raul, Raul's mother is Agnes. Who is Ana's grandmother} 
  		\end{itemize}
  		\item \textbf{Functional}: it is based on the evaluation of functions (like maths) recursively  (\textit{Lisp y Haskell}).
  		\begin{itemize}
  			\item Example: \textit{the factorial from 0 and 1 is 1 and n is the factorial from n * factorial (n-1). What is the factorial from 3?} 
  		\end{itemize}
  	\end{itemize}
\end{frame}

\begin{frame}{Programming paradigms types (II)}
	\begin{block}{Imperative programming}
		Describes, by a set of instructions that change the \alert{program state}, \alert{how} the task should be implemented.  
  	\end{block}
  	\begin{itemize}
  		%\item \textbf{Procedural}: organizes the program using collections of subroutines related by means of invocations (\textit{C, Python}).
  		%\begin{itemize}
  		%	\item Example: \textit{The cooking process consists of 20 lines of code. When it is used, it only calls the function (1 line).} 
  		%\end{itemize}
  		\item \textbf{Structural}: is based on nesting, loops, conditionals and subroutines. \texttt{GOTO} command is forbidden 				 \textit{(C, Pascal, Python)}.
  		\begin{itemize}
  			\item Example: \textit{reviewing products of a shopping list and add the item X to the shopping if it is available.} 
  		\end{itemize}
        \item \textbf{Object-Oriented Programming}
  	\end{itemize}
\end{frame}

\begin{frame}{Programming paradigms types (III)}{}
	\begin{block}{Object-Oriented Programming}
		Evolves from imperative programming. It is based on \alert{objects} that allow express the \alert{characteristics} and \alert{behavior} in a closer way to real life. 
  	\end{block}
  	\begin{itemize}
  		\item \textbf{Main characteristics}: abstraction, encapsulation, polymorphism, inheritance, modularity, etc.
		\item Example: \textit{a car has a set of properties (color, fuel type, model) and a functionality (speed up, shift gears, braking).} 
  	\end{itemize}
  	
\textit{\alert{There are many other paradigms such as Event-Driven programming, Concurrent, Reactive, Generic, etc.}}
\end{frame}

\begin{frame}{Programming paradigms types (IV)}{Classification}
    %\input{figs/paradigmas.tex}
	\includegraphics[scale=0.4]{figs/paradigmas}
	%\caption{http://www.computer.org/csdl/mags/it/2011/05/mit2011050030-abs.html}
	
	\centering Python supports the three major paradigms, although it stands out for the OOP
\end{frame}

\section[Object-Oriented Programming]{Object-Oriented Programming}

\subsection{Objectives}

\begin{frame}{Object-Oriented Programming}{Objectives}
\begin{itemize}
  	\item \textbf{Reusability}: Ability of software elements to serve for the construction of many different applications.
  	\item \textbf{Extensibility}: Ease of adapting software products to specification changes.
  	\item \textbf{Maintainability}: Amount of effort necessary for a product to maintain its normal functionality.   
  	\item \textbf{Usability}: Ease of using the tool.
 % 	\item \textbf{Robustness}: Ability of software systems to react appropriately to exceptional conditions.   
 % 	\item \textbf{Correction}: Ability of software products to perform their tasks accurately, as defined in their specifications.
  	\end{itemize} 	
\end{frame}

\subsection{Basic concepts}

\begin{frame}{Object-Oriented Programming}{Concepts (I)}
	\begin{block}{Class}
		 Generic entity that groups attributes and functions
  	\end{block}	


    \begin{columns}[t]
 	   \column{.5\textwidth}

	    \begin{block}{Atribute}
		Individual characteristics that determine the qualities of an object
  	    \end{block}	

		\begin{figure}
			\includegraphics[width=0.8\linewidth]{figs/clase}	
		\end{figure}				

 	   \column{.5\textwidth}

	    \begin{block}{Method}
		 Function responsible for performing operations
  	    \end{block}	
		\begin{figure}
			\includegraphics[width=0.8\linewidth]{figs/metodo}
		\end{figure}				
   \end{columns}
\end{frame}

\begin{frame}{Object-Oriented Programming}{Concepts (IV)}
	\vfill\begin{block}{Object or instance}
		 Specific representation of a class, namely, a class member with their corresponding attributes.
  	\end{block}	  	
		\begin{figure}
			\includegraphics[scale=0.5]{figs/instancia}\hfill
			\includegraphics[width=0.5\linewidth]{figs/laika}
		\end{figure}				
\end{frame}

\begin{frame}[plain]{Object-Oriented Programming}{Concepts (V)}
	\begin{center}
	\includegraphics[width=0.6\linewidth]{figs/pacman}\\
	\smallskip
	\tiny{\href{http://blog.sklambert.com/introduction-to-oop-for-game-development/}{(Source)}}
	\end{center}
\end{frame}

\begin{frame}[fragile]{Object-Oriented Programming}{Concepts (VI)}
	\centering {Two operations on classes}
    \begin{columns}
 	   \column{.50\textwidth}
	   		\begin{block}{Instantiation}
			Creates a new object\\
			Standard functional notation\\
			\bigskip
			\centering{\texttt{x = MyClass()}}\\
	   		\end{block}
	   		\begin{exampleblock}{Example}
\begin{verbatim}
time = Time()
\end{verbatim}
	   		\end{exampleblock}

 	   \column{.50\textwidth}
	   		\begin{block}{Attribute references}
			Accesses an attribute value\\
			Standard dot syntax\\
			\bigskip
			\centering{\texttt{obj.name}}\\
	   		\end{block}
	   		\begin{exampleblock}{Example}
\begin{verbatim}
time.hour = 4
print(time.hour)
hour = time.hour
\end{verbatim}
	   		\end{exampleblock}
	\end{columns}
\end{frame}

\subsection{Constructors}

\begin{frame}{Object-Oriented Programming}{Constructors (I)}
	\begin{block}{Constructor}
		 Method called when an object is created. It allows the initialization of attributes. 
  	\end{block}	
		\begin{figure}
			\includegraphics[scale=0.5]{figs/constructor}
		\end{figure}				
\end{frame}

\begin{frame}{Concepts of OOP}{Constructors (II)}
		Instantiation creates empty objects
		\begin{itemize}
			\item We usually need to initialize attributes
			\item Initialization operations
		\end{itemize}
		\alert{Constructor}: Method called when an object is created
		\begin{itemize}
			\item In Python, it is the \texttt{\_\_init\_\_()}
			\item A constructor can get arguments
		\end{itemize}
\end{frame}

\begin{frame}{Object-Oriented Programming}{Constructors (III)}
	\vspace{-0.3cm}
	\begin{exampleblock}{}
	\vspace{-0.3cm}
		\lstinputlisting{code/Time3.py}
	\end{exampleblock}
\end{frame}



%\begin{frame}{Object-Oriented Programming}{Synthesizing OOP terminology}
% 		  \begin{itemize}
%			\item \small{Software objects mimics physical objects.} % un  individuo o ejemplar de una clase 
%		   	\begin{itemize}
%				\item \footnotesize{An object contains \textit{attributes} (state) and a \textit{behaviour}}. % behavior u operaciones
%				\item \footnotesize{Example: A dog has a name (state) and may be a bit (behaviour).}
%		  	\end{itemize}
		  %	\item Objects are grouped into classes.
%		  	\item \small{A \alert{class} is a set of objects with common characteristics and behaviour.}
%		  	\item \small{An \alert{object} is called an \alert{Instance} of a class.}
		  	% It can also be described as a pattern or template that is used to create objects.
%			\item \textit{Members} of a class:
%		    	\begin{itemize}
%				\item \footnotesize{\alert{Properties}: Data describing an object.}
%				\item \footnotesize{\alert{Methods}: What an object can do.}%Something that can be done to object. % o procedimiento perteneciente a un objeto.
					% modo en que se comunican los objetos entre síi. 
%		  	    \end{itemize}
			% \item Property = Member, attribute, variable, etc.
%		  \end{itemize}
%\end{frame}

\begin{frame}[plain]%{Object-Oriented Programming}{Code example}
    \vspace{-0.3cm}

    \begin{columns}
 	   \column{.7\textwidth}
			\begin{exampleblock}{dogs.py}
			\vspace{-0.3cm} 
				\lstinputlisting[basicstyle=\ttfamily\scriptsize]{code/dogs.py}
			\end{exampleblock}

 	   \column{.3\textwidth}
			\begin{exampleblock}{Output}
			\vspace{-0.3cm} 
				\lstinputlisting[basicstyle=\ttfamily\scriptsize]{code/dogs-output.txt}
			\end{exampleblock}

		\href{https://gist.github.com/dfbarrero/9b30c749986885373c8c250b3901ec9d}{(Source code)}

        \bigskip

        \pause

        \centering UML class diagram

        \begin{tikzpicture}
        \umlclass{Dog}
            {+ name : str\\ + age : int} 
            {+ bit ()  :  void\\ +describe () : void}
        \end{tikzpicture}

    \end{columns}
\end{frame}

\subsection{Game example}

\begin{frame}[plain]{Object-Oriented Programming}{Game example}
	\begin{center}
    	\centering \includegraphics[width=0.8\linewidth]{figs/gameObjects}\\
		\smallskip
		\tiny{\href{http://blog.nuclex-games.com/2010/01/game-architecture-day-2/}{(Source)}}
	\end{center}
\end{frame}

\section{Inheritance}
\subsection{Definition}

\begin{frame}{Inheritance}{Definition}
\vspace{-0.2cm}
	\begin{block}{Inheritance}
	Mechanism of \alert{reusing} code in OOP. Consists of generating child classes from other existing (\alert{super-class}) allowing the use and adaptation of the attributes and methods of the parent class to the child class
  	\end{block}	

    \bigskip

		A subclass inherits all the attributes and methods from its superclass
        \begin{itemize}
		\item {\textit{Superclass}: ``Father'' of a class}
		\item {\textit{Subclass}: ``Child'' of a class}
        \end{itemize}
\end{frame}

\subsection{Examples}
\begin{frame}{Inheritance}{Examples of simple inheritance (I)}
    \begin{tikzpicture}
    \umlclass{Dog}
        {+ name : str\\ + age : int} 
        %{+ name : str\\ + age : int\\ + fleas : int} 
        {+ bit()  :  void\\ + describe() : void}

    \umlclass[x=3]{Cat}
        {+ name : str\\ + age : int} 
        {+ scratch()  :  void\\ + describe() : void}
    \end{tikzpicture}
\end{frame}

\begin{frame}{Inheritance}{Examples of simple inheritance (II)}
    \begin{tikzpicture}
    \umlclass{Animal}
        {+ name : str\\ + age : int} 
        {+ describe() : void}
    \umlclass[x=-2, y=-3]{Dog}
        %{+ fleas : int} 
        {} 
        {+ bit()  :  void}

    \umlclass[x=2, y=-3]{Cat}
        {} 
        {+ scratch() : void}

    \umlinherit[geometry=|-]{Dog}{Animal}
    \umlinherit[geometry=|-]{Cat}{Animal}
    \end{tikzpicture}
\end{frame}

\begin{frame}[plain]%{Characteristics}{Examples of simple inheritance (III)}
	\begin{exampleblock}{}
	\vspace{-0.3cm} 
		\lstinputlisting[basicstyle=\ttfamily\scriptsize]{code/animals.py}
	\end{exampleblock}

	\href{https://gist.github.com/dfbarrero/9b30c749986885373c8c250b3901ec9d}{(Source code)}
\end{frame}

\begin{frame}{Inheritance}{Examples of simple inheritance (III)}
	\textit{Class hierarchy}: A set of classes related by inheritance  \\
	\begin{figure}
		\includegraphics[scale=0.33]{figs/herencia1}
		\caption{{\scriptsize Example of simple Inheritance in OOP. Obtained from: \url{http://android.scenebeta.com}}}
	\end{figure}
\end{frame}

\subsection{Types of inheritance}

\begin{frame}{Inheritance}{Types of inheritance (I)}
	\begin{block}{Types of inheritance}
	\begin{itemize}
		\item If the child class inherits from a single class is called \alert{single inheritance}.
		\item if it inherits from more classes is \alert{multiple inheritance}.
	\end{itemize}
	\end{block}
	\medskip
	   \textit{Python allows both; simple and multiple inheritance.}
\end{frame}

\begin{frame}{Inheritance}{Types of inheritance (II)}
	\begin{figure}
		\includegraphics[scale=0.5]{figs/herencia2}
		\caption{{\scriptsize Example of multiple Inheritance in OOP. Obtained from: \url{http://www.avizora.com}}}
	\end{figure}
\end{frame}

\section{Concepts of OOP}
\subsection{Polymorphism}

\begin{frame}{Concepts of OOP}{Polymorphism (I)}
    \begin{columns}
 	   \column{.4\textwidth}

	\begin{block}{Polymorphism}
		Mechanism of object-oriented programming that allows to invoke a method whose implementation will depend on the object that does it.
  	\end{block}	
% TIPOS DE POLIMORFISMO
%		\begin{itemize}
%			\item \justifying\textbf{Polimorfismo de sobrecarga}: mismo nombre y tipos de 						  datos en diferentes clases.
%			\item \justifying\textbf{Polimorfismo paramétrico}: mismo nombre diferentes 							  tipos de datos. 
%			\item \justifying\textbf{Polimorfismo de subtipado}: Permite invocar un método 						  de una clase especializada a partir de la clase padre.
%		\end{itemize}
%	\begin{figure}

 	   \column{.6\textwidth}

    \begin{tikzpicture}
    \umlclass{Animal}
        {+ name : str\\ + age : int} 
        {+ describe() : void\\+ attack() : void}
    \umlclass[x=-2, y=-3]{Dog}
        %{+ fleas : int} 
        {} 
        {+ attack()  :  void}

    \umlclass[x=2, y=-3]{Cat}
        {} 
        {+ attack() : void}

    \umlinherit[geometry=|-]{Dog}{Animal}
    \umlinherit[geometry=|-]{Cat}{Animal}
    \end{tikzpicture}


    \end{columns}

	  %\vspace{-0.2cm}
%		\includegraphics[width=4.6cm]{figs/polimorfismo1}
%		\vspace{-0.2cm}
%		\caption{\scriptsize{Example of polymorphism. Obtained from: \url{http://virtual.uaeh.edu.mx}}}
%	\end{figure}
\end{frame}

%\begin{frame}{Characteristics}{Polymorphism (II)}
%	\begin{figure}
%		\includegraphics[scale=0.57]{figs/polimorfismo2}
%		\vspace{-0.1cm}
%		\caption{{\scriptsize Example of polymorphism. Obtained from: \url{http://datateca.unad.edu.co}}}
%	\end{figure}
%\end{frame}

\begin{frame}[plain]%{Characteristics}{Examples of simple inheritance (III)}
	\begin{exampleblock}{}
	\vspace{-0.3cm} 
		\lstinputlisting[basicstyle=\ttfamily\scriptsize]{code/animalsPolymorphism.py}
	\end{exampleblock}

	\href{https://gist.github.com/dfbarrero/0b401926b92a5e36ad08a3eae5fbd16c}{(Source code)}
\end{frame}

%\subsection{Abstraction}

%\begin{frame}{Concepts of OOP}{Abstraction}
%	\begin{block}{Abstraction}
%		Mechanism that allows the isolation of the not relevant information to a level of knowledge.
%  	\end{block}	
	
%	\begin{itemize}
%		\item \textit{A driver does not need to know how the carburetor works.} 
%		\item \textit{To talk on the phone does not need to know how the voice is transferred.} 
%		\item \textit{To use a computer do not need to know the internal composition of their materials}. 							  
%		
%%		\vfill\item \textit{\alert{En Python no existen de manera nativa las clases 					abstractas pero si mediante el módulo ABC \cite{PythonTeam}.}}
%	\end{itemize}
%\end{frame}

\subsection{Encapsulation}

\begin{frame}{Concepts of OOP}{Encapsulation (I)}
	\begin{block}{Encapsulation}
		Mechanism use to provide an access level to methods and attributes for avoiding unexpected state changes
  	\end{block}	

	\begin{figure}
		\includegraphics[scale=0.35]{figs/abstraccion2}
		\scriptsize \href{http://www.onlinebuff.com}{(Source)}
	\end{figure}
\end{frame}

\begin{frame}{Concepts of OOP}{Encapsulation (II)}
    The most common access levels are:
	
	\begin{itemize}
		\item \textbf{public}: visible for everyone  [default in Python]
		\item \textbf{private}: visible for the creator class [start with a double underscore and does not end in the same manner]
		\item \textbf{protected}: visible for the creator class and its descendents \alert{[does not exist in Python]}
	\end{itemize}

     Methods ``geters'' and ``setters'' to control the access to attributes
\end{frame}

\begin{frame}[plain]%{Characteristics}{Examples of simple inheritance (III)}
	\begin{exampleblock}{}
	\vspace{-0.3cm} 
		\lstinputlisting[basicstyle=\ttfamily\scriptsize]{code/access.py}
	\end{exampleblock}

	\href{https://gist.github.com/dfbarrero/a6ebc6632fb9fe06a2c8c6a47a465ab5}{(Source code)}
\end{frame}

%\section{Classes in Python}
% \subsection{Syntax}
%Sintaxis básica
%\begin{frame}{Nomeclatura de Python}{Sintaxis(I)}
%	\begin{itemize}
%		\item \justifying \textbf{Impresión por pantalla}: \texttt{print}. Por defecto solo admite la codificación inglesa (sin tildes ni ñ). 
%		\begin{itemize}
%			\item Para que las admita, se escribe antes: \# -*- coding: utf-8 -*-
%			\item Para imprimir texto se escribe entre comillas dobles.
%		\end{itemize}
%		\item \justifying\textbf{Indentación}: Python utiliza espacios o tabulaciones para definir los distintos niveles en el código, por lo tanto si un 													   fragmento se encuentra a la derecha del superior indica que está contenido en la instrucción anterior.
%		\item \justifying \textbf{Comentarios}: Comienzan con \#.
%	\end{itemize}
%    \begin{columns}
% 	   \column{0.8\textwidth}
%			\begin{block}{holaMundo.py}
%			\vspace{-0.3cm} 
%				\lstinputlisting[basicstyle=\ttfamily\scriptsize]{code/holaMundo.py} % contar elementos
%			\end{block}
%	\end{columns}			

%\end{frame}

%\begin{frame}{Nomeclatura de Python}{Sintaxis(II)}
%	\begin{itemize}
%		\item \justifying \textbf{Asignación}: Se expresa mediante un igual.
%		\item \justifying \textbf{Atributo}: sustantivo expresado con minúscula.
%		\item \justifying \textbf{Condiciones}: \texttt{if/elif/else + condición:}. Permiten expresar que hacer si se satisface o no una condición. Para 														expresar una condición de igualdad se utiliza \texttt{==}
%    \begin{columns}
% 	   \column{0.8\textwidth}
%			\begin{block}{semaforo.py}
%			\vspace{-0.3cm} 
%				\lstinputlisting[basicstyle=\ttfamily\scriptsize]{code/semaforo.py} % contar elementos
%			\end{block}
%	\end{columns}			
%	\end{itemize}
%\end{frame}


%\begin{frame}{Nomeclatura de Python}{Sintaxis(III)}
%	\begin{itemize}
%		\item \justifying \textbf{Estructuras iterativas}: \texttt{while/for + condición:}. Permiten expresar el mismo código mientras se cumpla la condición.	
%	\end{itemize}
%    \begin{columns}
% 	   \column{0.8\textwidth}
%			\begin{block}{provincias.py}
%			\vspace{-0.3cm} 
%				\lstinputlisting[basicstyle=\ttfamily\scriptsize]{code/provincias.py} % contar elementos
%			\end{block}
%	\end{columns}				 
%\end{frame}


%\begin{frame}{Classes in Python}{Syntax (I)}
%\vspace{-0.2cm}
%\begin{itemize}			
%		\item \small{\textbf{Class}: Start with the word \alert{class} followed by class name written in \alert{capital letter} and a colon [Substantives].}
%		\item \small{\textbf{Attributes}: A lowercase noun.}
%		\begin{itemize}
%		\item \footnotesize{There is no need to declare attributes.}
%		\end{itemize}
		
%		\item \small{\textbf{Inherited class}: Similar to a class but the class name followed by the class father in brackets.}
%		\item \small{\textbf{Instance}: Object in lower case followed by the class assignment.}
%			\vspace{-0.2cm} 
 % 	   \begin{columns}
% 	   \column{0.8\textwidth}
%			\begin{block}{\small{coche.py}}
%			\vspace{-0.3cm} 
%				\lstinputlisting[basicstyle=\ttfamily\scriptsize]{code/coche.py} % contar elementos
%				\vspace{-0.2cm} 
%			\end{block}
%	\end{columns}		
%\end{itemize}			
%\end{frame}

%\begin{frame}{Classes in Python}{Syntax (II)}
%\begin{itemize}
%		\item \textbf{Method}: Start with the word \alert{def}, and later the method, a verb, in lower case is written. Next, the parameter in brackets and a colon (\texttt{print\_name()}).
%   \begin{itemize}
%   \item Methods receive automatically a reference to the object (usually named \texttt{self}).
%   \end{itemize}
%		\item \textbf{Constructor}: Method whose name is \texttt{\_\_init\_\_()}, the first attribute is \texttt{self} and then the class attributes are written.

%		\item \textbf{main}: Method defined with \texttt{def main():}. In it, the wished commands are specified and after it, an exit condition is created. The \texttt{sys} module  is required to be imported at the beginning. 
%		\item All methods and attributes are public.
%			\begin{itemize}
%				\item By convention, private members begin with double underscore (\texttt{\_\_varName}, \texttt{\_\_method\_name()})
%			\end{itemize}
%\end{itemize}			
%\end{frame}


%\begin{frame}{Classes in Python}{Syntax (III). Example 1}
%    \begin{columns}
 %	   \column{0.8\textwidth}
%			\begin{block}{main.py}
%			\vspace{-0.3cm} 
%				\lstinputlisting[basicstyle=\ttfamily\scriptsize]{code/main.py} % contar elementos
%			\end{block}
%	\end{columns}				 	
%\end{frame}

%\begin{frame}{Classes in Python}{Syntax (IV). Example 2}
%\vspace{-0.2cm}
%    \begin{columns}
% 	   \column{0.8\textwidth}
%			\begin{block}{bicicleta.py}
%			\vspace{-0.3cm} 
%				\lstinputlisting[basicstyle=\ttfamily\scriptsize]{code/bicicleta.py} % contar elementos
%			\vspace{-0.3cm} 
%			\end{block}
%	\end{columns}				 	
%\end{frame}

%\begin{frame}{Classes in Python}{Syntax (V). Example 3}
%	\vspace{-0.2cm}
%	\begin{block}{Time.py}
%	\vspace{-0.3cm}
%		\lstinputlisting{code/Time.py}
%	\vspace{-0.3cm} 
%	\end{block}
%\end{frame}

\subsection{More about methods}

\begin{frame}{Concepts of OOP}{Other special methods}
In addition to special method \texttt{\_\_init\_\_}, there are several others, including:\\
	\begin{block}{}
		\vspace{-0.15cm}
		\begin{itemize}
		\item \footnotesize{\texttt{\_\_str\_\_(self)} It should return a string with \texttt{self} information.  When \texttt{print()} is invoked with the object, if the method \texttt{\_\_str\_\_()} is defined, Python shows the result of running this method on the object.}
		\item \footnotesize{\texttt{\_\_len\_\_(self)} It should return the length or ``size'' of object (number of elements if is a \textit{set} or \textit{queue}).}
		%\item \footnotesize{\texttt{\_\_add\_\_(self, otro\_obj)} It allows to apply the addition operator (+) to objects of the class in which it is defined.}
		%\item \footnotesize{\texttt{\_\_mul\_\_(self, otro\_obj)} It allows to apply the multiplication operator (*) to objects of the class in which it is defined.}
		%\item \footnotesize{\texttt{\_\_comp\_\_(self, otro\_obj)} It allows to apply the comparison operators (<, >, <=, >=, ==, !=) to objects of the class in which it is defined. It should return 0 if they are equal, -1 if \texttt {self} is smaller than \texttt{other\_obj} and 1 if \texttt{self} is greater than \texttt{other\_obj}.}
		\end{itemize}
		\vspace{-0.2cm}
	\end{block}	
\end{frame}

\subsection{Overriding methods}

\begin{frame}{Concepts of OOP}{Overriding methods (I)}

	Often we need to adapt an inheritanced method: \alert{Overriding}

	\begin{exampleblock}{Overriding example}
		\vspace{-0.2cm}
		\lstinputlisting{code/Overriding.py}
		\vspace{-0.2cm}
	\end{exampleblock}
\end{frame}
	
\begin{frame}{Concepts of OOP}{Overriding methods (II)}
		Still possible to get superclass' method with \texttt{super()}

	\begin{exampleblock}{\texttt{super()} example}
		\vspace{-0.2cm}
		\lstinputlisting{code/Overriding2.py}
		\vspace{-0.2cm}
	\end{exampleblock}
\end{frame}

\section{Arcade}

\begin{frame}[plain]
	\begin{exampleblock}{}
		\vspace{-0.2cm}
		\lstinputlisting{code/arcade-1.py}
		\vspace{-0.2cm}
	\end{exampleblock}
\end{frame}

\begin{frame}[plain]
	\begin{exampleblock}{}
		\vspace{-0.2cm}
		\lstinputlisting{code/arcade-2.py}
		\vspace{-0.2cm}
	\end{exampleblock}
\end{frame}

\begin{frame}{Arcade}
	The \texttt{arcade.Window} class.

	\begin{block}{}
		\vspace{-0.15cm}
		\begin{itemize}
		\item \footnotesize{\texttt{on\_draw()}}. Override this function to add your custom drawing code
		\item \footnotesize{\texttt{on\_update(delta\_time: float)}}. Move everything. Perform collision checks. Do all the game logic here
		\item \footnotesize{\texttt{on\_key\_release(symbol: int, modifiers: int)}} 
		\item \footnotesize{\texttt{on\_mouse\_release(x: float, y: float, button: int, modifiers: int)}}. Override this function to add mouse button functionality
		\item \footnotesize{\texttt{set\_viewport(left: float, right: float, bottom: float, top: float)}}. Set the coordinates we can see
		\end{itemize}
		\vspace{-0.2cm}
	\end{block}	

	Check out \href{https://api.arcade.academy/en/stable/api/window.html#arcade-window}{(reference documentation)}
\end{frame}

\section{Exercises}
\subsection{Exercise 1: Asteroids}
	\begin{frame}{Exercise 1: Asteroids}
	\vspace{-0.3cm}
    \begin{columns}
 	   \column{.6\textwidth}
		\centering \includegraphics[width=\linewidth]{figs/asteroids.png}\\
		\tiny{\href{http://gamedevelopment.tutsplus.com/tutorials/quick-tip-intro-to-object-oriented-programming-for-game-development--gamedev-1805}{(Source)}}
		\column{.4\textwidth}
	\begin{enumerate}
	\item Identify the classes in the Asteroids videogame
	\item Identify attributes contained in the previous classes
	\item Identify methods contained in the previous classes
	\end{enumerate}
	\end{columns}
\end{frame}

\subsection{Exercise 2: Tetris}
	\begin{frame}{Exercise 2: Tetris}
    \begin{columns}
 	   \column{.4\textwidth}
		\centering \includegraphics[width=\linewidth]{figs/tetris.png}\\
		\tiny{\href{http://gamedevelopment.tutsplus.com/tutorials/quick-tip-intro-to-object-oriented-programming-for-game-development--gamedev-1805}{(Source)}}
		\column{.4\textwidth}
	\begin{enumerate}
	\item Identify the classes in the Tetris videogame
	\item Identify attributes contained in the previous classes
	\item Identify methods contained in the previous classes
	\end{enumerate}
	\end{columns}
\end{frame}

\subsection{Exercise 3: Pac-Man}
	\begin{frame}{Exercise 3: Pac-Man}
    \begin{columns}
 	   \column{.5\textwidth}
		\centering \includegraphics[width=\linewidth]{figs/pacman2.png}\\
		\tiny{\href{http://gamedevelopment.tutsplus.com/tutorials/quick-tip-intro-to-object-oriented-programming-for-game-development--gamedev-1805}{(Source)}}
		\column{.5\textwidth}
	\begin{enumerate}
	\item Identify the classes in the Pac-Man videogame
	\item Identify attributes contained in the previous classes
	\item Identify methods contained in the previous classes
	\end{enumerate}
	\end{columns}
\end{frame}




%\begin{frame}{Exercise statement}{\texttt{Animal} class}
%	\begin{enumerate}
%		\item Create the \texttt{animal} class. 
%		\item Create the constructor. The class will have the attributes \texttt{tipo} and \texttt{patas}.
%		%con las variables tipo de animal y número de 		patas.
%		\item Create the get methods from both attributes which receive like own parameter the animal through \texttt{self} and return respectively the \texttt{tipo} and \texttt{patas}. 
%		\item Create two instances of animals using the constructor.
%		\item Print the attributes of both instances.
%	\end{enumerate}	
%\end{frame}

%\begin{frame}{Solved exercise}{\texttt{Animal} class}
%	\vspace{-0.2cm}
%    \begin{columns}
% 	   \column{0.8\textwidth}
%			\begin{block}{animales.py}
%			\vspace{-0.3cm} 
%				\lstinputlisting[basicstyle=\ttfamily\scriptsize]{code/animales.py} % contar elementos
%			\vspace{-0.3cm} 
%			\end{block}
%	\end{columns}
%\end{frame}

%\begin{frame}{Solved exercise}{\texttt{Animal} class}
%	\begin{enumerate}
%		\item Create a \texttt{gato} class in the same file which inherits from the \texttt{animal} class. 
%		\item Create the constructor and add the \texttt{sonido} attribute. 
%		\item Create the method \texttt{maullar} which prints the sound MIAU. 
%		\item Create a instance and check the methods.
%	\end{enumerate}	
%\end{frame}

%\begin{frame}{Solved exercise}{Class \texttt{Animals}}
%	\vspace{-0.4cm}
%    \begin{columns}
% 	   \column{0.8\textwidth}
%			\begin{block}{animales.py}
%			\vspace{-0.3cm} 
%				\lstinputlisting[basicstyle=\ttfamily\scriptsize]{code/gato.py} % contar elementos
%			\vspace{-0.3cm} 
%			\end{block}
%	\end{columns}
%\end{frame}
% añadir otro ejemplo de clases GIS , pickle de clases y más referencias
%\begin{frame}{Exercise statement}{Class \texttt{Parcela}}
%	\begin{enumerate}
%		\item Create a script containing the class \texttt{Parcela}. 
%		\item Create the constructor. The class will have the attributes \texttt{uso\_suelo} and \texttt{valor}.
%		%con las variables tipo de animal y número de 		patas.
%		\item Create the \texttt{valoracion} method to calculate the tax associated with the parcel as follows: 
%		\begin{itemize}
%		\item For single-family residential: \textit{tasa = 0.05 * valor}
%		\item For multifamily residential: \textit{tasa = 0.04 * valor}
%		\item For all other land uses: \textit{tasa = 0.02 * valor}
%		\end{itemize}
%		\item Use the class from another \textit{script} named \texttt{tasaparcela.py} which you create una instance of \texttt{Parcela} named \texttt{miparcela} using the constructor.
%		\item Print the attribute  \texttt{uso\_suelo} of the instance.
%		\item Use the method \texttt{valoracion} of \texttt{Parcel} to calculate the assessment of \texttt{miparcela}.
%	\end{enumerate}	
%\end{frame}

%\begin{frame}{Solved exercise}{Class \texttt{Parcela}}
%	\vspace{-0.26cm}
%  %  \begin{columns}
% %	   \column{0.8\textwidth}
%			\begin{block}{claseparcela.py}
%			\vspace{-0.3cm} 
%				\lstinputlisting[basicstyle=\ttfamily\scriptsize]{code/claseparcela.py} % contar elementos
%			\vspace{-0.3cm} 
%			\end{block}
%%	\end{columns}
%%	\tiny{\href{http://esripress.esri.com/display/index.cfm?fuseaction=display&websiteID=276&moduleID=0}{Source}}
%\end{frame}

%\begin{frame}{Solved exercise}{Use of \texttt{Parcela}}
%	\vspace{-0.4cm}
%    \begin{columns}
% 	   \column{0.8\textwidth}
%			\begin{block}{tasaparcela.py%}
%			\vspace{-0.3cm} 
%				\lstinputlisting[basicstyle=\ttfamily\scriptsize]{code/tasaparcela.py} % contar elementos
%			\vspace{-0.3cm} 
%			\end{block}
%	\end{columns}
%		\tiny{\href{http://esripress.esri.com/display/index.cfm?fuseaction=display&websiteID=276&moduleID=0}{Source}}
%\end{frame}

%\begin{frame}[plain]{Solved exercise. Serializando objetos \texttt{Parcela}}{}
%%	\vspace{-0.4cm}
%  %  \begin{columns}
% %	   \column{0.8\textwidth}
%			\begin{block}{\footnotesize{tasaparcela\_pickle.py}}
%			\vspace{-0.2cm} 
%				\lstinputlisting[basicstyle=\ttfamily\scriptsize]{code/tasaparcela_pickle.py} % contar elementos
%			\vspace{-0.2cm} 
%			\end{block}
%%	\end{columns}
	
%\end{frame}

%\subsection{Approach to a final problem}
%\begin{frame}{Exercise statement}{\texttt{Rio} class}
%\vspace{-0.24cm}
%	\begin{enumerate}
%		\item Create the \texttt{Rio} class. 
%		\item Create the constructor and add the \texttt{nombre} and \texttt{longitud} attributes.
%		\item \texttt{Longitud} attribute must be private.
%		\item Create the \texttt{setLongitud} method which receives self and \texttt{longitudR} and allows the set of any value for \texttt{longitud}.
%		\item Create the \texttt{getNombre} method which obtains the name of the river.
%		\item Create the \texttt{getLongitud} method which obtains the river length.		 
%		\item Create an instance and check the methods.
%		\item Try to do an assignment of \texttt{rio.nombre} and other assignment with  \texttt{rio.longitud} What happen? It is correct  to invoke the method named \texttt{rio.getLongitud()} out of the classes? How do you explain that?
%	\end{enumerate}	
%\end{frame}

%\begin{frame}{Exercise statement}{Establishment of hierarchies from \texttt{Rio} class}
%\vspace{-0.24cm}
%	\begin{enumerate}
%		\item Add to the \texttt{Rio} class the attribute \texttt{caudal} and the method \texttt{trasvasar} which receives two rivers and transfers 5 liters from the first to the second.
%		\item Create the \texttt{Afluente} class which inherits from \texttt{Rio}.		
%		\item Create the method \texttt{\_\_init\_\_} of \texttt{Afluente} which initializes its \texttt{nombre} and \texttt{longitud} and, also,  \texttt{afluenteDeRio}, new attribute initialized with the name of the river which the affluent starts.
%		\item Is there any polymorphism in this sample?
%		\item Create the main and exit condition and try it. Does the main position affect to the application? 
%		\item Experiment now with conditions and iterative structures limiting when a river can transfer water or try to do some transfer at the same time.

%	\end{enumerate}	
%\end{frame}

\end{document}
