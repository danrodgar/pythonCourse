%%%%%%%%%%%%%%%%%%%%%%%%%%%%%%%%%%%%%%%%%%%%%%%%%%%%%%%%%%%%%%%%%
% MUW Presentation
% LaTeX Template
% Version 1.0 (27/12/2016)
%
% License:
% CC BY-NC-SA 4.0 (http://creativecommons.org/licenses/by-nc-sa/3.0/)
%
% Created by:
% Nicolas Ballarini, CeMSIIS, Medical University of Vienna
% nicoballarini@gmail.com
% http://statistics.msi.meduniwien.ac.at/
%
% Customized for UAH by:
% David F. Barrero, Departamento de Automática, UAH
%%%%%%%%%%%%%%%%%%%%%%%%%%%%%%%%%%%%%%%%%%%%%%%%%%%%%%%%%%%%%%%%%

\documentclass[10pt,compress]{beamer} % Change 10pt to make fonts of a different size
\mode<presentation>

\usepackage[english,spanish]{babel}
\usepackage{fontspec}
\usepackage{tikz}
\usepackage{etoolbox}
\usepackage{xcolor}
\usepackage{xstring}
\usepackage{listings}
\usepackage{tikz}
\usetikzlibrary{matrix,chains,positioning,decorations.pathreplacing,arrows,shapes}

\usetheme{UAH}
\usecolortheme{UAH}
\setbeamertemplate{navigation symbols}{} 
\setbeamertemplate{caption}[numbered]

%%%%%%%%%%%%%%%%%%%%%%%%%%%%%%%%%%%%%%%%%%%%%%%%%%%%%%%%%%%%%%%%%
%% Presentation Info
\title[Modules]{Modules}
\author{\asignatura\\\carrera}
\institute{}
\date{}
%%%%%%%%%%%%%%%%%%%%%%%%%%%%%%%%%%%%%%%%%%%%%%%%%%%%%%%%%%%%%%%%%


%%%%%%%%%%%%%%%%%%%%%%%%%%%%%%%%%%%%%%%%%%%%%%%%%%%%%%%%%%%%%%%%%
%% Descomentar para habilitar barra de navegación superior
\setNavigation
%%%%%%%%%%%%%%%%%%%%%%%%%%%%%%%%%%%%%%%%%%%%%%%%%%%%%%%%%%%%%%%%%

%%%%%%%%%%%%%%%%%%%%%%%%%%%%%%%%%%%%%%%%%%%%%%%%%%%%%%%%%%%%%%%%%
%% Configuración de logotipos en portada
%% Opacidad de los logotipos
% \newcommand{\opacidad}{1}
%% Descomentar para habilitar logotipo en pié de página de portada
% \renewcommand{\logoUno}{Images/isg.png}
%% Descomentar para habilitar logotipo en pié de página de portada
%\renewcommand{\logoDos}{Images/CCLogo.png}
%% Descomentar para habilitar logotipo en pié de página de portada
%\renewcommand{\logoTres}{Images/ALogo.png}
%% Descomentar para habilitar logotipo en pié de página de portada
%\renewcommand{\logoCuatro}{Images/ELogo.png}
%%%%%%%%%%%%%%%%%%%%%%%%%%%%%%%%%%%%%%%%%%%%%%%%%%%%%%%%%%%%%%%%%

%%%%%%%%%%%%%%%%%%%%%%%%%%%%%%%%%%%%%%%%%%%%%%%%%%%%%%%%%%%%%%%%%
%% FOOTLINE
%% Comment/Uncomment the following blocks to modify the footline
%% content in the body slides. 


%% Option A: Title and institute
\footlineA
%% Option B: Author and institute
%\footlineB
%% Option C: Title, Author and institute
%\footlineC
%%%%%%%%%%%%%%%%%%%%%%%%%%%%%%%%%%%%%%%%%%%%%%%%%%%%%%%%%%%%%%%%%

\begin{document}

%%%%%%%%%%%%%%%%%%%%%%%%%%%%%%%%%%%%%%%%%%%%%%%%%%%%%%%%%%%%%%%%%
% Use this block for a blue title slide with modified footline
{\titlepageBlue
    \begin{frame}
        \titlepage
    \end{frame}
}

\institute{\asignatura}

\begin{frame}[plain]{}
	\begin{block}{Objectives}
		\begin{enumerate}
		\item Understand the relevance to use modules and packages.
		\item Be able to install some widely used Python packages% about  geospatial software.
		\item Be able to apply some modules and packages of both Python Standard Library% and others Python Geospatial Libraries.
		\end{enumerate}
	\end{block}
\end{frame}

{
\disableNavigation{white}
\begin{frame}[shrink]{Table of Contents}
 \frametitle{Table of Contents}
 \tableofcontents
  % You might wish to add the option [pausesections]
\end{frame}
}

\section{Introduction}
\begin{frame}{Introduction (I)}
		You loose everything when exit the interpreter
			\begin{itemize}
			\item \textit{Solution}: Write it down in a script
			\end{itemize}
		When a script becomes big, it is difficult to maintain
			\begin{itemize}
			\item \textit{Solution}: Split your script in several ones
			\end{itemize}
		As you get more scripts, you will need to reuse your functions
			\begin{itemize}
			\item \textit{Solution}: Create a \alert{module}
			\item \textbf{Module}: A file that contains definitions, functions and classes
			\end{itemize}
		If a module is too big, it is too difficult to maintain
			\begin{itemize}
			\item \textit{Solution}: Create a \alert{package}
			\item \textbf{Package}: A module of modules
			\end{itemize}
\end{frame}

\begin{frame}{Introduction (II)}
		\begin{block}{Why modules?}
			\begin{itemize}
			\item \textbf{Main function}: Organization.
			\item \textbf{Reuse}: To provide software solutions, that have been proven to work, to solve similar problems.
			\end{itemize}
		\end{block}
		
\end{frame}

\section{Modules}

\subsection{Using modules}
\begin{frame}{Using modules}{Creation and Implementation}
	\vspace{-0.2cm}
	A module is just a Python script with \texttt{.py} extension
	\vspace{-0.2cm}
	%creo que está mal fibo.py
	\begin{exampleblock}{fibo.py}
	\vspace{-0.2cm}
	\lstinputlisting[basicstyle=\ttfamily\scriptsize,numbers=left]{code/fibo.py}
	\vspace{-0.2cm}
	\end{exampleblock}
\end{frame}

\begin{frame}{Using modules}{Where is it stored?}

 Accessible and reusable module:
 \begin{itemize}
 \item  Set path in the file directory where the module is stored.
 \item Variable \texttt{PYTHONPATH}
 \end{itemize}
 \end{frame}
 
\begin{frame}[fragile]{Using modules}{How do I use them? (I)}
	\begin{block}{}
	\begin{verbatim}
>>> import fibo
>>> fibo.fib(1000)
1 1 2 3 5 8 13 21 34 55 89 144 233 377 610 987
>>> fibo.fib2(100)
[1, 1, 2, 3, 5, 8, 13, 21, 34, 55, 89]
>>> fibo.__name__
'fibo'
>>> fib = fibo.fib
>>> fib(100)
1 1 2 3 5 13 21 34 55 89
\end{verbatim}
	\vspace{-0.2cm}
	\end{block}
\end{frame}

\begin{frame}[fragile]{Using modules}{How do I use them?  (II)}
	A module can import other modules
		\begin{itemize}
		\item Name conflicts may arise: Each module has a symbol table
		\item It means you should invoke it as \texttt{modname.itemname}
		\end{itemize}
 	It is possible to import items directly
		\begin{itemize}
		\item \texttt{from module import name1, name2}
		\item \texttt{from module import *}
		\item It uses the global symbol table (no need to use the modname)
			\end{itemize}
	
	\begin{exampleblock}{}
	\begin{verbatim}
>>> from fibo import fib, fib2
>>> fib(100)
1 1 2 3 5 8 13 21 34 55 89 144 233 377 610 987
\end{verbatim}
	\end{exampleblock}
\end{frame}

\begin{frame}[fragile]{Using modules}{How do I use them?  (III)}
	\vspace{-0.2cm}
	\begin{block}{List zip file contents (file.zip must exist. Open in read mode)}
	\vspace{-0.2cm}
	\lstinputlisting[numbers=left]{code/zip.py}
	\vspace{-0.2cm}
	\end{block}
	
	\vspace{-0.2cm}
	\centering \footnotesize{Several examples here: \url{http://pymotw.com/2/PyMOTW-1.132.pdf}}
\end{frame}

\begin{frame}[fragile]{Using modules}{How do I use them?  (IV)}
	%\vspace{-0.2cm}
	Error while importing:
	\begin{itemize}
	\item The module does not exist.
	\item The module name has not been well written.
	\item The module is not on the search path of Python modules:
	\begin{enumerate}
	\item By default, it searches in the current directory.
	\item If it does not find it here, it then searches in the directories of the environment variable \texttt{PYTHONPATH}.
	\begin{itemize}
	\item \texttt{echo \$PYTHONPATH} (from Linux/Windows console)
	\item \texttt{import sys}\\
	\texttt{print(sys.path)}
	\end{itemize}
	\item If it still does not find, it then searches in the installation directories of Python.
	\end{enumerate}
	\end{itemize}
    Warning: \texttt{PYTHONPATH} is not \texttt{PATH}
\end{frame} 
	
\subsection{Executing modules}

\begin{frame}{Executing modules}{Modules as scripts (I)}
	When a module is imported, its statements are executed
		\begin{itemize}
		\item It declares functions, classes, variables ...
		\item ... and also executes code
		\item It serves to initialize the module
		\end{itemize}
	Very useful to use modules as programs and libraries
\end{frame}

\begin{frame}[shrink,plain,fragile]{Executing modules}{Modules as scripts (II)}
    \small
	\vspace{-0.3cm}
	\begin{exampleblock}{fibo2.py}
	\vspace{-0.2cm}
	\lstinputlisting[numbers=left]{code/fibo2.py}
	\vspace{-0.2cm}
	\end{exampleblock}

	\vspace{-0.5cm}
	\begin{columns}
 	   \column{0.5\textwidth}
	\begin{exampleblock}{}
	\vspace{-0.2cm}
	\begin{verbatim}
    (In Linux console)
$ python3 fibo2.py 50
1 1 2 3 5 8 13 21 34
\end{verbatim}
	\vspace{-0.2cm}
	\end{exampleblock}

 	   \column{0.5\textwidth}
	\begin{exampleblock}{}
	\vspace{-0.2cm}
	\begin{verbatim}
    (In Python interpreter)
>>> import fibo2
>>> fibo2.fib(50)
1 1 2 3 5 8 13 21 34
\end{verbatim}
	\vspace{-0.2cm}
	\end{exampleblock}

    \end{columns}
\end{frame}

\subsection{Compiled Python files}
\begin{frame}{Compiled Python files}{}
	We said Python is an interpreted language
		\begin{itemize}
		\item ... this is almost a lie
		\end{itemize}
	Python, as other interpreted languages, has a speed-up trick
		\begin{itemize}
		\item It can use bytecode, just as Java
		\end{itemize}
	\textbf{Bytecode}: Intermediate code between machine code and source code
		\begin{itemize}
		\item Faster than source code, slower than machine code.
		\item It is transparent to the programmer.
		\item The first time a \texttt{.py} file is executed, it is compiled automatically, generating a \texttt{.pyc} file.
		\end{itemize}
\end{frame}

\subsection{Content of a module}

\begin{frame}[fragile]{Content of a module}{The dir() function}
	Very usefull to get an insight to a module
	\begin{itemize}
		\item It returns the names defined in a module
		\item Without arguments, it returns your names
	\end{itemize}
	\begin{exampleblock}{}
	\begin{verbatim}
>>> import fibo, sys
>>> dir(fibo)
['__name__', 'fib', 'fib2']
>>> dir()
['__builtins__', ... , '__spec__']
>>> variable = 'Hello'
>>> dir()
['__builtins__', ... , '__spec__', 'variable']
\end{verbatim}
	\vspace{-0.2cm}
	\end{exampleblock}
\end{frame}

\section{Packages}
\subsection{Package concept}
\begin{frame}{Packages}{Package concept (I)}
		If a module gets too big, many problems arise
			\begin{itemize}
			\item Name collisions
			\item It is good to organize modules in a bigger structure: \textit{Packages}
			\end{itemize}
		Packages can be seen as ``dotted module names''
			\begin{itemize}
			\item It is just a module that contains more modules
			\item Make life easier in big proyects
			\item The name \texttt{A.B} designates a submodule \texttt{B} in a package named \texttt{A}
			\end{itemize}
		Must contain a \texttt{\_\_init\_\_.py} file in the root directory
			\begin{itemize}
			\item Executed when the package is imported for the first time
			\end{itemize}
\end{frame}

\begin{frame}[shrink,plain]{Packages}{Package concept (II)}
	\centering \textit{Sound module structure}
	\lstinputlisting{code/module.txt}
\end{frame}

\subsection{Importing a package}
\begin{frame}{Packages}{Importing a package (I)}
	\centering{\alert{Ways to use a package}}\\
	\bigskip
	\begin{flushleft}
	Import an individual module
		\begin{itemize}
		\item \texttt{import sound.effects.echo}
		\item Use function as \texttt{sound.effects.echo.echofilter(input, output)}
		\end{itemize}
	Alternative way to import an individual module
		\begin{itemize}
		\item \texttt{from sound.effects import echo}
		\item Use function as \texttt{echo.echofilter(input, output)}
		\end{itemize}
	Alternative way to import an individual module
		\begin{itemize}
		\item \texttt{from sound.effects.echo import echofilter}
		\item Use function as \texttt{echofilter(input, output)}
		\end{itemize}
	\end{flushleft}
\end{frame}

\begin{frame}{Packages}{Importing a package (II)}
	Imagine we run \texttt{from sound import *}
		\begin{itemize}
		\item In theory, it would import the whole package
		\item In practice, it would take too much time
		\end{itemize}
	There is a convention to avoid waste of resources
		\begin{itemize}
		\item There may be a variable \texttt{\_\_all\_\_} defined in \texttt{\_\_init\_\_}
		\item \texttt{\_\_all\_\_} contains modules to be imported
		\end{itemize}

	\begin{block}{sounds/effects/\_\_init\_\_.py}
	\vspace{-0.2cm}
	\lstinputlisting{code/package.py}
	\vspace{-0.2cm}
	\end{block}
\end{frame}

\subsection{Installing packages}
\begin{frame}[fragile]{Packages}{Installing packages}
	Command-line automatic tool: \texttt{pip} (sometimes \texttt{pip3})
		\begin{itemize}
			\item Very similar to \texttt{apt-get} in Linux
		\end{itemize}

	\begin{block}{pip usage (from OS terminal)}
	\vspace{-0.2cm}
	\begin{verbatim}
$ python -m pip install SomePackage
\end{verbatim}
or
	\begin{verbatim}
$ pip install SomePackage
\end{verbatim}
	\vspace{-0.2cm}
\end{block}

	\begin{exampleblock}{}
	\vspace{-0.2cm}
	\begin{verbatim}
$ pip install Pillow
\end{verbatim}
	\vspace{-0.2cm}
\end{exampleblock}
\end{frame}

\subsection{What has been developed about Python packages?}
\begin{frame}{Packages}{What has been developed?}
\centering \includegraphics[scale=0.267]{figs/PyPI.pdf}\\
\end{frame}


\section{The Python Standard Library}

\subsection{\texttt{os} module}

\begin{frame}{\texttt{os} module}{Functions to manipulate files and processes}
\vspace{-0.2cm}
\footnotesize{
\begin{block}{}
\vspace{-0.15cm}

\begin{itemize}			
\item \textbf{Functions for managing files and paths}: \texttt{dir(os.path)}
\item \textbf{Create directories}. Example: \texttt{os.mkdir('data')}
\item \textbf{Current working directory}: \texttt{os.getcwd()}
\item \textbf{Moving to a certaing directory}. Example: \texttt{os.chdir('data')}
\item \textbf{Value of an environment variable}. Example: \texttt{os.chdir(os.environ['HOME'])}
\item \textbf{Rename a file}. Example: \texttt{os.rename('fich1.py', 'palindrome.py')}
\item \textbf{Deleting a file}. Example: \texttt{os.remove('practica1.py')}
\item \textbf{List the files in the current directory}. Example: \texttt{os.listdir(os.curdir)}
\item \textbf{List the files in a certain directory}. Example: \texttt{os.listdir('c:$\backslash\backslash$data')}
\item \textbf{Call operating system (execute OS services)}. Example: \texttt{os.kill}, \texttt{os.execv}, etc.
\vspace{-0.1cm}
\end{itemize}

\end{block}	
}

Warning!: Linux and Windows use different path separator (\texttt{os.path.sep})
    \begin{itemize}
        \item Linux: \texttt{myscripts/script.py}
        \item Windows: \texttt{myscripts\textbackslash script.py}
    \end{itemize}

\end{frame}

\subsection{\texttt{sys} module}

\begin{frame}{\texttt{sys} module}
\vspace{-0.2cm}
It provides access to some variables maintained by the interpreter at run-time.\\
\begin{block}{}
\footnotesize{
\begin{itemize}			
\item \textbf{List the arguments passed to \textit{script} on the command line}: \texttt{sys.argv}
\item \textbf{Python output}. \texttt{sys.exit}
\item \textbf{Files for access to input, output and standard error of the interpreter}: \texttt{sys.stdin}, \texttt{sys.stdout}, \texttt{sys.stderr}, respectively. 

\end{itemize}
}
\end{block}	
\end{frame}

\begin{frame}{\texttt{sys} module}{Example}

\vspace{-0.2cm}
	\begin{block}{example\_sys.py}

	\lstinputlisting{code/example_sys.py}

	\end{block}
\end{frame}

\subsection{\texttt{time} module}

\begin{frame}{\texttt{time} module (I)}

\begin{itemize}
\item It provides functions related to the measurement of time.
\item Python provides the date and time of three ways:\\
\begin{itemize}
\item Tuple: year-month-day-hour-min-sec-dayweek-day year-x (\textit{tup})
\item String (\textit{str})
\item Total of seconds since an origin (\textit{sec})
\end{itemize}
\end{itemize}

\end{frame}

\begin{frame}{\texttt{time} module (II)}

\begin{block}{}
\footnotesize{
\begin{itemize}			
\item \textbf{Current time}: \texttt{time()}
\item \textbf{Time elapsed since the start of the execution}. \texttt{process\_time()}
\item \textbf{Pause n seconds}. \texttt{sleep()}
\item \textbf{GMT}. \texttt{gmtime()}
\item \textbf{Local time}. \texttt{localtime()}
\item \textbf{Convert the tuple to a character string}. \texttt{asctime()}
\item \textbf{Convert the tuple to a string}. \texttt{strftime()}
\item \textbf{Convert the tuple to seconds}. \texttt{mktime()}
\item \textbf{Convert the seconds to a string}. \texttt{ctime()}
\item \textbf{Convert the string to a tuple}. \texttt{strptime()}
\item \ldots
\end{itemize}
}
\end{block}	
\end{frame}

%\begin{frame}{\texttt{time} module (III)}{Example}

%\vspace{-0.2cm}
%	\begin{block}{example\_time.py}

%	\lstinputlisting{code/time_formatted.py}

%	\end{block}
%Output:\\
%\small{\texttt{Local current time : time.struct_time(tm_year=2020, tm_mon=3, tm_mday=9, tm_hour=18, tm_min=16, tm_sec=49, tm_wday=0, tm_yday=69, tm_isdst=0)}}
%\end{frame}



\section{Other cool code examples}

\subsection{Example 1: Open a web browser}
\begin{frame}{Cool code examples}{Example 1: Open a web browser}
	\vspace{-0.2cm}
	\begin{block}{browser.py}
	\vspace{-0.2cm}
	\lstinputlisting{code/brower.py}
	\vspace{-0.2cm}
	\end{block}
\end{frame}

\subsection{Example 2: Create a thumbnail}
\begin{frame}[plain]{Cool code examples}{Example 2: Create a thumbnail}
	\begin{columns}
 	   \column{.60\textwidth}

		\vspace{-0.2cm}
		\begin{block}{thumbnail.py}
		\vspace{-0.2cm}
		\lstinputlisting{code/thumbnail.py}
		\vspace{-0.2cm}
		\end{block}

		

  		\column{.50\textwidth}
		\vspace{-0.2cm}
		\centering \includegraphics[width=\linewidth]{figs/africa.jpg}\\
	 	\texttt{africa.jpg}
	\end{columns}
		\vspace{-0.2cm}
	\centering \tiny{\href{http://www.pythonforbeginners.com/gui/how-to-use-pillow}{(Source)}}
\end{frame}

\subsection{Example 3: List a directory contents}
\begin{frame}{Cool code examples}{Example 3: List a directory contents}
		\vspace{-0.2cm}
		\begin{block}{dir.py}
		\vspace{-0.2cm}
		\lstinputlisting{code/dir.py}
		\vspace{-0.2cm}
		\end{block}

		\vspace{-0.2cm}
	\centering \tiny{\href{http://www.pythonforbeginners.com/code-snippets-source-code/having-fun-with-os-walk-in-python/}{(Source)}}
\end{frame}

\subsection{Example 4: Send an email with Gmail}
\begin{frame}{Cool code examples}{Example 4: Send an email with Gmail}
	\begin{columns}
 	   \column{1.1\textwidth}
		\vspace{-0.2cm}
		\begin{block}{gmail.py}
		\vspace{-0.2cm}
		\lstinputlisting{code/gmail.py}
		\end{block}
	\end{columns}

	\centering \tiny{\href{http://www.pythonforbeginners.com/code-snippets-source-code/using-python-to-send-email/}{(Source)}}
\end{frame}

\appendix
\section<Bibliographic references>*{\appendixname}
\subsection<Bibliographic references>*{Bibliographic references}

\begin{frame}[plain,allowframebreaks]
  \frametitle<presentation>{Bibliographic references}

  \begin{thebibliography}{2}
  
  \beamertemplatebookbibitems
  % libro
   \bibitem{vanRosum}[van Rosum, 2012]
    G. van Rossum, Jr. Fred L. Drake.
    \newblock \emph{Python Tutorial Release 3.2.3, chapter 6}.
    \newblock Python Software Foundation, 2012. 
  % libro
   \bibitem{Lutz}[Lutz, 2013]
    M. Lutz.
    \newblock \emph{Learning Python}.
    \newblock O'Reilly, 2013.
    
     \bibitem{Bahit}[Bahit, 2008]
     E. Bahit.
    \newblock \emph{Curso: Python para principiantes}.
    \newblock Creative Commons Atribución-NoComercial 3.0, 2012.
 \newpage
    
    \bibitem{vanRosum}[vanRosum, 2012]
    G. van Rossum, Jr. Fred L. Drake.
    \newblock \emph{The Python Library Reference. Release 3.2.3}.
    \newblock Python Software Foundation, 2012. 
    
    \bibitem{Hellman}[Hellman, 2011]
     D. Hellman.
    \newblock \emph{The Python Standard Library by Example (Developer's Library)}.
    \newblock Addison Wesley Professional, 2011.
    
    % \bibitem{Swaroop}[Swaroop, 2008]
   %  C. H. Swaroop.
   % \newblock \emph{A byte of Python}.
   % \newblock Creative Commons Atribucion-NoComercial 3.0, 2008.
 

 
  \end{thebibliography}

\end{frame}



\end{document}
